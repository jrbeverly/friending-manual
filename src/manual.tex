\documentclass[a4paper,11pt,titlepage]{scrartcl}
\batchmode

\usepackage[utf8]{inputenc}
\usepackage[T1]{fontenc} % LY1 also works
\usepackage{textcomp} % to get the right copyright, etc.
\usepackage[lining,tabular]{fbb} % so math uses tabular lining figures
\usepackage[scaled=.95,type1]{cabin} % sans serif in style of Gill Sans
\usepackage[varqu,varl]{zi4}% inconsolata typewriter
\useosf % change normal text to use proportional oldstyle figures

\usepackage{microtype}
\usepackage{graphicx}
\usepackage{enumitem}

\usepackage{listings}
\lstset{basicstyle=\ttfamily,frame=single,xleftmargin=3em,xrightmargin=3em}
\usepackage[os=win]{menukeys}
\usepackage{framed}
\usepackage{etoolbox}
\AtBeginEnvironment{leftbar}{\sffamily\small}

\usetikzlibrary{chains,arrows,shapes,positioning}
\usepackage[hidelinks]{hyperref}

%%%%
%%%%
%%%%

\newcommand\setDefaultMenuColor{\renewmenucolortheme{gray}{RGB}{70,127,215}{255,255,255}{255,255,255}}
\newcommand\setNavigationMenuColor{\renewmenucolortheme{gray}{RGB}{230,230,230}{120,144,156}{98,106,110}}
\renewmenumacro{\keys}{angularkeys}
\setDefaultMenuColor

%%%%
%%%%
%%%%

\newcommand{\textapp}[1]{{\fontfamily{cmss}\selectfont#1}}
\newcommand{\textuser}[1]{{\fontfamily{crm}\selectfont\textit{#1}}}
\newcommand{\textvar}[1]{{\textsc{#1}}}
\newcommand{\textcomp}[1]{{\fontfamily{pnc}\selectfont#1}}
\newcommand{\textui}[1]{{\fontfamily{pag}\selectfont#1}}
\newcommand{\textaction}[1]{{\fontfamily{cmtt}\selectfont#1}}
\newcommand{\textmath}[1]{{\fontfamily{zplm}\selectfont#1}}

%%%%
%%%%
%%%%

\newcommand{\Friending}{\textapp{Friending}}
\newcommand{\Screenshot}[1]{\textui{#1}}
\newcommand{\action}[1]{\textaction{#1}}
\newcommand{\abbrevation}[1]{\texttt{#1}}
\newcommand{\gterm}[2]{\item #1 - #2}

%%%%
%%%%
%%%%


%%%%
%%%%
%%%%

\newcommand\appSignUp{Sign Up}
%%%%
%%%%
%%%%

%% NEW COMMAND
%% name of screenshot, caption, label
\newcommand{\appscreenshot}[3]{
	\begin{figure}[h!]%	
	\includegraphics[width=4.4cm]{{"../assets/screenshots/#1"}.png}%
	\centering%
	\caption{#2}%
	\label{#3}%
	\end{figure}%
}
%%

\newcommand{\appbutton}[1]{
	\includegraphics[height=1em]{{"../assets/buttons/#1"}.png}%
}

\newcommand{\quicknav}{\Screenshot{Navigation Menu} (Figure~\ref{fig:navigation})}

%%%%
%%%% 
%%%%

\begin{document}

\begin{titlepage}
	\centering
	\includegraphics[width=100mm]{{"../assets/icons/friending-doc"}.png}\\
	\vspace{1.5cm}
	{\huge\bfseries Friending User Guide\par}
	\vspace{2cm}
	{\Large\itshape Jonathan Beverly (\href{https://jrbeverly.gitlab.io/}{jrbeverly})\par}
	\vfill
	{\large\textbf{Abstract}\par}
	\begin{paragraph} 
	\Friending is an online dating, friendship, and social networking website that features member-created quizzes and multiple-choice questions.  The application is available for review at \href{https://jrbeverly-friending.gitlab.io/friending/}{Friending}. 
	\end{paragraph}
	\\\vspace{0.5cm}
	{\large \today\par}
\end{titlepage}

\clearpage
\tableofcontents

\clearpage
\section{List of Figures}
\listoffigures

% Manual
%
% The beginning of the manual

\clearpage
\section{Introduction}
\label{sec:introduction}

\subsection{Product overview}
\label{sec:overview}

\Friending{} is an online dating, friendship, and social networking mobile application that features user-created questionnaires and multiple choice questions.  \Friending{} has two primary features: joining groups to find people similar to you or registering for events happening in your local area.

You can create a group around one of your interests, then define questionnaires that can be used to match other members of the group.  You can join these groups and fill in questionnaires to be matched with members of the group.  The questionnaire answers are used to determine a mutual match based on the your and others responses.  You will receive a notification if a mutual match is found.

Events are managed by an event host that controls the event.  Each event has a questionnaire that you will fill out to join.  You can invite people to join your event, and when you are ready start the processing of pairing up people in the event.    The participants of your event will then be matched into pairs.  You will receive a notification when a match is set.

\subsection{Home Pages}
\label{sec:homepages}

When opening \Friending{} for the first time you will be presented with \Screenshot{Start} (Figure~\ref{fig:start}).   If you have already signed in before then you will be taken to \Screenshot{Sign In} (Figure~\ref{fig:signin}):

\appscreenshot{Start}{Start}{fig:start}

\Screenshot{Start} is the opening page for \Friending{}. It allows you to learn more about \Friending{} through a carousel.  Simply \action{tap} on the \keys{Start the tour} button, and you will be taken to \Screenshot{Onboarding} (Figure~\ref{fig:onboarding}).

\appscreenshot{Onboarding}{Onboarding}{fig:onboarding}

You can \action{Swipe Left} or \action{Swipe Right} to move through the descriptions to learn more about \Friending.  When you are ready to begin, \action{Tap} on the \keys{Continue} button.  You will be taken to \Screenshot{Sign Up} (Figure~\ref{fig:signup}).

When you have successfully created your account or logged in, you will be presented with the \Screenshot{Home} (Figure~\ref{fig:home}):

\appscreenshot{HomePage (none)}{The first run home page}{fig:homestart}

When you are in the application you can use the \Screenshot{Navigation Menu} (Figure~\ref{fig:navigation}) to quickly navigate throughout the application.  \action{Tap} on the \appbutton{menu} button to open the \Screenshot{Navigation Menu}.   The \Screenshot{Navigation Menu} is accessible on top level pages such as but not limited to: \Screenshot{Home} (Figure~\ref{fig:home}), \Screenshot{Groups} (Figure~\ref{fig:home}) and \Screenshot{My Questionnaires} (Figure~\ref{fig:home}).

\appscreenshot{Navigation Menu}{Navigation Menu}{fig:navigation}

The links will take you to the following locations:

\begin{itemize}
\item Profile $\rightarrow$ Profile Settings (Figure 7.1).
\item Preferences $\rightarrow$ Profile Settings (Figure 7.2).
\item Notifications $\rightarrow$ Notifications (Figure 10.5).
\item Events $\rightarrow$ Parties (Figure 9.3).
\item Groups $\rightarrow$ Community Search (Figure 8.1).
\item Questionnaires $\rightarrow$ Community Templates (Figure 11.4).
\item Billing $\rightarrow$ Settings (Figure 6.1).
\item Settings $\rightarrow$ Settings (Figure 5.3).
\item Sign out $\rightarrow$ Start Page (Figure 1.1).
\end{itemize}

\appscreenshot{Homepage}{Home}{fig:home}

\clearpage
\section{Conventions}
\label{sec:conventions}

\subsection{User assumptions}
\label{sec:assumptions}
You, the user of \Friending{}, are assumed to know the basics of touch-enabled applications.  If not, please refer to a user’s manual of your mobile device.  \Friending{} is only accessible through an internet-enabled mobile device.

\subsection{Use Cases}
\label{sec:usecases}
A use case is.  All the manipulations in a use case are based on the assumption that \Friending{} is running, and you are signed in to your account unless otherwise specified.  For Sections ~\ref{sec:signup} and ~\ref{sec:signin}, you are assumed to not be signed in.  

\subsection{Notational conventions}
\label{sec:notational}
The text conventions below are used in this manual:

\begin{itemize}
\item \textapp{Computer Modern Sans Serif} is used for program names and file names.
\item Computer Modern Roman is used for normal text.
\item \textuser{Times New Roman Italics} is used for key users of the software.
\item \textmath{Palatino} is used for mathematical formulas.
\item \textvar{Palatino Small Capitals} is used for key name variables 
\item \textcomp{Avant Garde} is used for key components of the software.
\item \textui{Avant Garde Bold} is used for user interface components of the software.
\item \textaction{Computer Modern Typewriter} is used for physical actions taken by the user when interacting with the user interface of the software.
\end{itemize}

\subsection{Visual conventions}
\label{sec:visual}
The images below are used throughout the manual:

\begin{itemize}
\item \appbutton{editting} is used to indicate a field that can be toggled for editing.
\item \appbutton{calendar} is used to indicate a field that is editable and uses a calendar widget.
\item \appbutton{menu} is used to indicate a pull out menu.
\item is used to indicate the pages of a carousel menu.
\item \setDefaultMenuColor \keys{Sample}  is used to denote buttons of the user interface.
\item \setNavigationMenuColor \keys{Sample} is used to denote menu buttons of the user interface.
\end{itemize}
\setDefaultMenuColor

\subsection{Glossary of terms}
\label{sec:glossary}
The terms below are used through the manual:

\begin{itemize}
\gterm{\Friending{}}{The name of the application.}
\gterm{User}{The person who uses \Friending{}, addressed by “you”.}
\gterm{Member}{A user who joins a group.}
\gterm{Participant}{A user who fills out a response to a questionnaire.}

\gterm{Internet-Enabled Mobile Device}{A small computing device capable of accessible the internet.}
\gterm{One Time Password}{A password that is valid for a single sign on.}
\gterm{Approachable}{The trait of being approachable.}

\gterm{Fee}{A charge for \Friending{} services.}
\gterm{Billed}{A statement of money owed for \Friending{} services.}

\gterm{Profile}{The outline of "you", the user.}
\gterm{Event}{A one time event where multiple participants fill out questionnaires to receive matches with other participants in the event}
\gterm{Event Host}{A participant who initiates a event}
\gterm{Group}{A collection of members and questionnaires administered by a Group Administrator.}
\gterm{Group Administrator}{The administrator for a group that has questionnaires and members.}
\gterm{Questionnaire}{A set of questions to be answered by the participant.}
\gterm{Notification}{An email sent to a user.}

\gterm{Match}{A pairing of exactly two participants.}
\gterm{Mutual Match}{A match between two participants who have a compatibility score equal to or above the minimum match threshold.} 
\gterm{Minimum Match Threshold}{A percentage point at which the compatibility score between two participants is sufficient for a mutual match.}
\gterm{Compatibility Score}{The percentage compatibility between two participants.}
\gterm{Comparison Score}{The numeric value representing the similarity between two values.}
\gterm{Balanced Comparison Score}{A numeric value equalling $85\%$ of the best possible comparison score for a questionnaire.}
\gterm{Question Match Criteria}{The criteria determining the comparison score between two question answers.}

\gterm{Query}{A request for information from \Friending{}.}

\gterm{Number Pad}{A grid of numbers used for inputting numeric values.}

\gterm{Heterosexual}{Is romantic attraction, sexual attraction or sexual behavior between persons of the opposite sex or gender.}
\gterm{Bisexual}{Is romantic love or sexual attraction toward both males and females}
\gterm{Gay}{A male homosexual a male who experiences romantic love or sexual attraction to other males.}
\gterm{Lesbian}{A female homosexual a female who experiences romantic love or sexual attraction to other females.}
\end{itemize}

\subsection{Abbreviations}
\label{sec:abbreviations}
\begin{itemize}
\item G\&SD - Gender and Sexual Diversities
\item GUI - Graphic User Interface
\item SMS - Short Messaging Service
\end{itemize}

\subsection{Basic User interface and user interactions goals}
\label{sec:goals}
\Friending{} is inspired by mail-in oriented dating services that appeared in the early stages of "online" dating.  These computer dating services operate by having you fill out a paper questionnaire which will be mailed in with a nominal fee.  The questionnaires are geared towards people seeking a date.  

\Friending{} makes use of the internet, removing the slow and sometimes unreliable mail service.  You can fill out, submit and view your responses to questions faster than using postal mail.  Additionally \Friending{} includes more communal oriented features than the early concepts, such as enabling you to host your own events with questionnaires.

The basic goals with the user interface of \Friending{} is the following:
\begin{itemize}
\item Designing a template should be approachable due to simplified rating systems. 
\item Each view should have an obvious feature and method of navigation.
\item There is no need to require a time commitment; \Friending{} works in the background.
\end{itemize}

\subsection{Organization of this manual}
\label{sec:organization}
The remainder of this manual is organized based on use cases.  All Sections assume that you are already logged into \Friending{} with the exception of Sections ~\ref{sec:signup} and ~\ref{sec:signin}. This manual also contains troubleshooting in Section ~\ref{sec:troubleshootandtips} and  gives the limitations of the current version of \Friending{} in Section ~\ref{sec:limitations}.

\clearpage
\section{Account}
\label{sec:account}

\subsection{Signing up for your account}
\label{sec:signup}
To start using \Friending, create an account by opening the mobile application, \Friending. You will be presented with \Screenshot{Start} as shown in Figure~\ref{fig:start}. \action{Tap} on the \keys{Start the tour} button.

\appscreenshot{Sign Up}{Sign Up}{fig:signup}

Then, on \Screenshot{Sign Up} (Figure~\ref{fig:signup}), \action{enter} your \textvar{email}, \textvar{name} and \textvar{password}.  You should review the \textcomp{Terms of Service} and \textcomp{Privacy Policy} that define the expectations for you when using \Friending. Afterwards, \action{tap} on the \keys{Sign Up} button.

\subsection{Signing in to your account}
\label{sec:signin}
On \Screenshot{Start} (Figure~\ref{fig:signin}), tap the \keys{Sign In} button to start the sign in process.

\appscreenshot{Sign In}{Sign In}{fig:signin}

On \Screenshot{Sign In} (Figure~\ref{fig:signup}), \action{enter} your \textvar{email} and \textvar{password}.  Then, \action{tap} on the \keys{Sign In} button.  If you have forgotten your password, see section ~\ref{sec:reset}. 
\subsection{Viewing your settings}
\label{sec:settings}
To view your settings, \action{tap} on the \keys{Settings} button in the \Screenshot{Navigation Menu} (Figure~\ref{fig:navigation}). Once on \Screenshot{Settings} (Figure~\ref{fig:settings}), you will be able to review your settings. 

\appscreenshot{Settings}{Settings}{fig:settings}
\subsection{Updating your account details}
\label{sec:account-update}
To update your account details, \action{tap} on the \keys{Settings} button in the \Screenshot{Navigation Menu} (Figure~\ref{fig:navigation}).  \action{Tap} on the \keys{Account} button as shown in (Figure~\ref{fig:navigation}).

Once on the Settings (Figure 3.4), \action{enter} your changes to \textvar{email}, \textvar{phone number} and \textvar{password} fields. After you are satisfied with your changes, \action{tap} on the \keys{Save} button to save your changes.

\subsection{Resetting your password}
\label{sec:reset}
To reset your password, \action{tap} the \keys{Sign In} button on \Screenshot{Start} (Figure~\ref{fig:start}).

Then, \action{enter} your \textvar{email address} in the \textcomp{Email Address} field as in ~\ref{fig:reset}. Then tap on the \keys{Reset} link. An email will be sent your email address with a one-time password.  This password will allow you to sign in to your account, but will not change your password. The one-time-password has an expiration period of one (1) hour, when it expires, it will no longer allow access to your account.

You are encouraged to change your password after signing in with the one-time password.

\appscreenshot{Reset}{Reset your password}{fig:reset}
\subsection{Deleting your account}
\label{sec:account-delete}
\appscreenshot{Delete Account}{Delete account}{fig:accountdelete}
To delete your account, \action{tap} on the \keys{Settings} button on the \Screenshot{Navigation Menu} (Figure~\ref{fig:navigation}).

Then, while on \Screenshot{Settings} (Figure~\ref{fig:settings}), \action{tap} on the \keys{Delete Account} button.  You must agree to the deletion of your account in the \Screenshot{Confirmation Popup} (Figure~\ref{fig:accountdelete}). Your account will be deactivated and you will not receive any further matches or notifications during this period.  An email will be sent to notify you of your account deactivation and impending deletion.  Within 14 days your account will be permanently deleted along with all associated information.  If you sign in to your account within 14 days, your deletion request will be cancelled.  There is no way to recover a deleted account.

\clearpage
\section{Billing}
\label{sec:billing}

\subsection{Viewing your billing information}
\label{sec:billing-view}
\appscreenshot{Billing}{Billing}{fig:billing}
To view your account billing information, \action{tap} on the \keys{Settings} button on the \quicknav.
Then, while on \Screenshot{Settings} (Figure~\ref{fig:settings}), \action{tap} on \keys{Billing} button, you will be presented with your current billing information (Figure~\ref{fig:billing}).   To download a copy of an invoice, \action{tap} on the \appbutton{invoice} button.  This will download a copy of an invoice to your device.  If any billing information is incorrect, see Section ~\ref{sec:troubleshoot} on troubleshooting.

\subsection{Update your billing information}
\label{sec:billing-update}

\appscreenshot{Credit Information}{Credit Information}{fig:credit}
To update account billing information, \action{tap} on the \keys{Settings} button on the \quicknav.  Then, while on \Screenshot{Settings} (Figure~\ref{fig:settings}), \action{tap} on the \keys{Billing} button, you will be presented with your current billing information (Figure~\ref{fig:billing}). 
 
Then, while on \Screenshot{Billing} (Figure~\ref{fig:billing}), \action{tap} on the \appbutton{creditcard} button, you will be presented with your current credit card (Figure~\ref{fig:credit}).  \action{Enter} your changes to \textvar{First Name}, \textvar{Last Name}, \textvar{Card Number}, \textvar{Zip Code} and \textvar{Expiration Date} fields.  After you are satisfied with your changes, \action{tap} on the \keys{Save} button to save your changes.  

You will be emailed a confirmation that your billing information has been changed.  

\subsection{Downloading your invoices}
\label{sec:billing-invoices}
\appscreenshot{Billing Notifcation Pop}{Billing Notifcation}{fig:billingnotify}

\subsection{Paying for your account services}
\label{sec:billing-paying}
Paying for your account happens automatically.   You will need to set your billing information to use \Friending{}.   If \Friending{} is unable to bill your account, you will be unable to use \Friending{} until your billing information is corrected.  You will be billed on a monthly basis for your account.

You can see your billing information as specified in section ~\ref{sec:billing-view}.

\clearpage
\section{Profile}
\label{sec:profile}

\subsection{Updating your profile}
\label{sec:profile-update}

\appscreenshot{Profile}{Profile}{fig:profile}
To update your personal profile information, \action{tap} on the \keys{Profile} button on the \quicknav. Then, while \Screenshot{Profile} (Figure~\ref{fig:profile}), \action{enter} your changes to \textvar{name}, \textvar{birthday}, \textvar{gender}, \textvar{orientation} and \textvar{description} fields.  You can tap on the \textcomp{Profile Image} to set your profile picture.  After you are satisfied with your changes, \action{tap} on the \keys{Save} button to save your changes.  
\\\\
The \textcomp{gender} field is limited to a binary choice of male and female.  The \textcomp{orientation} field is limited to Heterosexual, Bisexual, Homosexual.  See section ~\ref{sec:limitations} on the limitations of \Friending{} with respect to G\&SD.

\subsection{Setting your notification preferences}
\label{sec:profile-notify}
\appscreenshot{Preferences}{Preferences}{fig:preferences}
To view your notification preferences, \action{tap} on the \keys{Settings} button in the \quicknav. Then, while on \Screenshot{Settings} (Figure~\ref{fig:settings}),  \action{tap} on the \keys{Notifications} button, you will be presented with your notification settings.  On \Screenshot{Preferences} (Figure~\ref{fig:preferences}) you can see the types of notifications that \Friending{} will send. To toggle email notifications for a notification type, \action{tap} on the check box.  If email notifications is enabled, the email icon will appear filled.  If email notifications is disabled, the email icon will appear empty. After you are satisfied with your changes, \action{tap} on the \keys{Save} button to save your changes.  Email notifications cannot be disabled for billing statements and account notifications.

\clearpage
\section{Groups}
\label{sec:groups}
\appscreenshot{Fill}{Group Questionnaire}{fig:fill}
\appscreenshot{Filling}{Group Questionnaire Answering}{fig:filling}
\appscreenshot{Group Edit}{Group Editting}{fig:groupedit}
\appscreenshot{Group Join}{Group Join}{fig:groupjoin}
\appscreenshot{Group}{Group View}{fig:group}
\appscreenshot{Groups}{Groups}{fig:groups}
\appscreenshot{Answer}{Submitted Questionnaire}{fig:answer}
\appscreenshot{Answers}{Questionnaire Answers}{fig:answers}
\subsection{Viewing groups}
\label{sec:groups-view}

\subsection{Searching groups}
\label{sec:groups-search}

\subsection{Joining a group}
\label{sec:groups-join}
To join a group, \action{tap} on the \keys{Group} button on the \quicknav.  Then, while on the \Screenshot{Groups} (Figure~\ref{fig:groups}), \action{enter} a \textvar{search query} into the search bar, you will be presented with groups matching your query.  Select the \textcomp{group} that matches your query to view \Screenshot{Group Details} (Figure~\ref{fig:group}).

Then, while on the \Screenshot{Group Details} page, \action{tap} on the \keys{Join} button.  The group will be added to your \Screenshot{Home Page} (Figure~\ref{fig:home}).   

\clearpage
\section{Events}
\label{sec:events}
\appscreenshot{Event Edit}{Event Edit}{fig:eventedit}
\appscreenshot{Event Fill}{Event Fill}{fig:eventfill}
\appscreenshot{Event Filled In}{Event Filled In}{fig:eventfilled}
\appscreenshot{Event Filling}{Event Filling}{fig:eventfilling}
\appscreenshot{Event Joined}{Event Joined}{fig:eventjoined}
\appscreenshot{Event}{Event}{fig:event}
\appscreenshot{Events}{Events}{fig:events}

\subsection{Joining an event}
\label{sec:events-join}
The only way to join a event is to be invited to a event.  To view your invitations, \action{tap} on the \keys{Notifications} button on the \quicknav. On \Screenshot{Notifications} (Figure~\ref{fig:navigation}) you can view all notifications for your account.  \action{Tap} on a event invitation notification to open the \Screenshot{Event View} (Figure~\ref{fig:home}).  

On the \Screenshot{Event View} page you can tap on the \keys{Join} button to join the event. This will mark your intentions to join the event - it does not guarantee that you will successfully join the event. See section ~\ref{sec:event-wait} for more details.   You may not be able to join a event based on your gender or sexual orientation, see section ~\ref{sec:event-req} for more details. You will receive a notification of your intention to join the event, visibile in your \Screenshot{Notifications} (Figure~\ref{fig:navigation}).   When you have successfully joined the event, you will see the event on your \Screenshot{Home} (Figure~\ref{fig:home}), in \Screenshot{Events} (Figure~\ref{fig:events}), and a notification in your \Screenshot{Notifications} (Figure~\ref{fig:notifications}).  

If you no longer wish to join the event, you can cancel your pending request to join by revisiting the \Screenshot{Event View} page.  \action{Tap} the \keys{Pending} button to cancel your pending request (Figure~\ref{fig:eventjoined}).  Once you have successfully joined a event, you cannot leave the event.

\subsection{Waiting to join an event}
\label{sec:events-wait}
When trying to join a event, if the event does not have sufficient participants for gender and sexual orientation matches, then you will be placed in a waiting list.  The waiting list is first come first serve.  As a event ensures that individuals will be matched with someone, you will remain in the waiting list until someone requests to join the event that has the potential to match with you. 

An example of this would be a heterosexual event.  This event requires an equal number of male and female participants to ensure that everyone will be matched with someone.  A male user requesting to join would be placed in a waiting list until a female user requests to join the event.  For a homosexual event, the event requires an even number of participants to ensure everyone is matched.  As such, a user requesting to join would be placed in the waiting list until another user requests to join the event.

When the event host start matching for the event, the waiting list will be cancelled and everyone on the list will be denied from joining the event.  If you do not successfully join the event, you will see a notification in your \Screenshot{Notifications}.

\subsection{Event Join Requirements}
\label{sec:events-req}
A event is defined with an orientation that defines how participants in a event are matched.  The currently defined orientations are Heterosexual, Bisexual, Gay, and Lesbian.  This means that a event with a Lesbian orientation would only accept you if you are female.  All participants in a event must match with someone in the event, as such the orientation restricts the participants to those that can match within the event.  See section ~\ref{sec:limitations} on the limitations of \Friending{} with respect to G\&SD.

\subsection{Viewing events}
\label{sec:events-view}
To view your events, \action{tap} on the \keys{Events} button on the \quicknav. This brings up the Events page (Figure~\ref{fig:event}). On the Events page (Figure~\ref{fig:events}) you can view all events that you have created, are waiting to join or are currently participating in.  \action{Tap} on the event to view the details of a event.  This opens the \Screenshot{Event} page (Figure~\ref{fig:event}).  

\subsection{Starting matching for events}
\label{sec:events-starting}
To host an event, \action{tap} on the \keys{Create Party} button on the \Screenshot{Home} (Figure~\ref{fig:home}). This opens the \Screenshot{Event Create} (Figure~\ref{fig:eventedit}). 

On the Event Create page you can set the properties of the event.  \action{Enter} your changes to \textvar{Title}, \textvar{Capacity}, \textvar{Orientation}, and \textvar{Description} fields.  The \textcomp{Orientation} field is limited to Heterosexual, Bisexual, Gay, and Lesbian.  To set the questionnaire, see section ~\ref{sec:questionnaire-create}.  After you are satisfied with your changes, \action{tap} on the Create button to create your event.   You can now see your event on your \Screenshot{Home} (Figure~\ref{fig:home}) or on the Events page (Figure~\ref{fig:events}).

\clearpage
\section{Group}
\label{sec:group}

\subsection{Viewing a group}
\label{sec:group-view}
To view a group, \action{tap} on the \textcomp{group} that you wish to view on the \Screenshot{Home Page} (Figure~\ref{fig:home}). This opens the \Screenshot{Group Details} (Figure~\ref{fig:group}). Then, while on the \Screenshot{Group Details}, \action{tap} on the \keys{View} button.  You will be presented with your \Screenshot{Group Home} (Figure~\ref{fig:home}).

\subsection{Leaving a group}
\label{sec:group-leave}
To leave a group, \action{tap} on the community that you wish to leave on \Screenshot{Home} (Figure~\ref{fig:home}). This opens the \Screenshot{Group Details} (Figure~\ref{fig:group}).  Then, while on \Screenshot{Group Details}, \action{tap} on the \keys{Leave} button.  The group will be removed from your \Screenshot{Home} (Figure~\ref{fig:home}).  If you wish to re-join the group, see section \ref{sec:group-join}.

\subsection{Creating your group}
\label{sec:group-create}
You can define your own group to discover yourself and others.  Within each group, you can create questionnaires that your members can fill out.  

To create a group, \action{tap} on the create group button on \Screenshot{Home} (Figure~\ref{fig:home}).  This opens \Screenshot{Group Details} (Figure ~\ref{fig:group}).  You can edit the values as defined in section ~\ref{sec:group-update}.  When you are satisfied with your values, you can save the group, which creates the group.  The group will then be accessible from your \Screenshot{Home} (Figure~\ref{fig:home}) or the \Screenshot{Group Search} (Figure~\ref{fig:home}).

\subsection{Updating your group}
\label{sec:group-update}
To edit a group you have created, navigate to \Screenshot{Group Admin} (Figure~\ref{fig:groupedit}) as described in section ~\ref{sec:group-view}.

Then, \action{tap} on the \keys{Manage} button and edit icons will appear next to the \textvar{Name} and \textvar{Description} fields (Figure~\ref{fig:groupedit}). These two fields are now editable. \action{Tap} on the field which you wish to edit and edit the value.  When you are satisfied with your changes, \action{tap} the \keys{Save} button.

\subsection{Viewing your group questionnaires}
\label{sec:group-questionnaire}

\subsection{Updating your group questionnaires}
\label{sec:group-questionnaire-edit}

\clearpage
\section{Questionnaires}
\label{sec:questionnaires}
\appscreenshot{Question Matching 1}{Question Matching 1}{fig:questionmatch1}
\appscreenshot{Question Matching 2}{Question Matching 2}{fig:questionmatch2}
\appscreenshot{Question-Editing}{Question-Editing}{fig:questionedit}
\appscreenshot{Questionnaire-Details}{Questionnaire-Details}{fig:questiondetails}
\appscreenshot{Questionnaire-Questions}{Questionnaire-Questions}{fig:questions}
\appscreenshot{Questionnaires}{Questionnaires}{fig:questionnaires}
\appscreenshot{Question-Weighting}{Question-Weighting}{fig:weighting}
\appscreenshot{Categories}{Categories}{fig:categories}
\subsection{Filling out a questionnaire}
\label{sec:questionnaire-fill}
To fill out a questionnaire, tap on the questionnaire on the Group Home page (Figure 3.13). This opens the Questionnaire Overview page (Figure 3.14). 

On the Questionnaire Overview page you can view all questions in the questionnaire.  Each question is marked with a solid   to indicate an answered question or empty heart   if you have not.  All questions must be answered before you can submit.  If you leave the page before finishing the questionnaire, your previous answers will be saved. To answer a question tap on a question.  You can tap any question in the questionnaire, in any order.  After you are satisfied with your answers, tap on the Submit button to submit your questionnaire.  

\Friending{} currently supports three types of questions: true-false, multiple choice, and multiple answer.  Depending on the type of question, you will be presented with one of the three pages:

To fill out a response for any question type, tap on the circle next to the answer of your choice.
You fill in an answer for yourself and for your ideal match, under You and Them columns respectively.  After you are satisfied with your responses, tap on the Save button to save your responses for this question.

\subsection{Flagging important questions}
\label{sec:questionnaire-importance}
To rate the importance of the question shown in Figure 3.15, tap on the star rating  under the label Importance.  The star which you tap will determine the rating out of 3.  The more stars, the higher degree of  importance to you the question is.  This degree of importance will be factored into the matchmaking process.  The default rating is 1 star. 

\subsection{Deleting a draft questionnaire}
\label{sec:questionnaire-delete}
You cannot delete a questionnaire that has not been submitted.   You are only able to submit one set of answers for a questionnaire.  You can submit the questionnaire (see section 3.2.3), then delete the submitted questionnaire.

If you delete your account, the draft questionnaire information will be deleted.

\subsection{Viewing your submitted questionnaire}
\label{sec:questionnaire-submitted-view}
To view a submitted questionnaire, tap on the questionnaire on the Group Home page (Figure 3.13). This opens the Questionnaire Overview page (Figure 3.14).   On the Submitted Questionnaire Overview page (Figure 3.16) you can view all questions in the questionnaire.  To view a question, tap on a question.  You can tap any question in the questionnaire, in any order.  

\subsection{Updating your submitted questionnaire}
\label{sec:questionnaire-submitted-update}
To update a submitted questionnaire, tap on the questionnaire on the Group Home page (Figure 3.13). This opens the Submitted Questionnaire Overview page (Figure 3.16).  On the Questionnaire Overview page you can view all questions in the questionnaire.  To view a question tap on a question.  You can tap any question in the questionnaire, in any order.  

\Friending{} currently supports three types of questions: true-false, multiple choice, and multiple answer (Figure 3.15).  To update a response for any question type, tap on the circle next to the answer of your choice.  You fill in an answer for yourself and for your ideal match, under You and Them columns respectively.   To update the importance of a question, follow the process defined under section 3.2.4.

After you are satisfied with your updates, tap on the Save button to save your updates for this question. 

\subsection{Removing your submitted questionnaire}
\label{sec:questionnaire-submitted-delete}
To delete a submitted questionnaire, tap on the questionnaire on the Group Home page (Figure 3.13). On the Submitted Questionnaire Overview page (Figure 3.16) you can view all questions in the questionnaire.  To delete a submitted questionnaire, select the Delete button.  The submitted questionnaire cannot be recovered after deletion.

\subsection{Viewing your match notifications}
\label{sec:questionnaire-matches}

To view your match notification, tap on the Notifications button on the Navigation Menu (Figure 1.2). This brings up the Notifications page (Figure 3.17).  On the Notifications page you can view all notifications for your account.  Tap on the match found notification to open the Match Profile View page (Figure 3.18).  

On the Match Profile View page you can then contact the match through email.  At present \Friending{} only supports contacting matches through email.

\appscreenshot{Match}{Match}{fig:match}
\appscreenshot{Notifications}{Notifications}{fig:notifications}

\clearpage
\section{Questionnaire Designs}
\label{sec:admin}

\subsection{Creating a questionnaire}
\label{sec:questionnaire-create}
On the Party Create page (Figure 3.22) you set the questionnaire.  Tap the Select Template Field to open the Template Select page (Figure 3.23).  On the Template Select page you can view all templates you have created.  Tap on the template to select the template for your party.  You will be returned to the Party Create page with the template set.

\subsection{Sending invitiations}
\label{sec:questionnaire-invite}
To view your events, tap on the Events button on the Navigation Menu (Figure 1.2). This brings up the Events page (Figure 3.21).   On the Events page (Figure 3.21) you can view all events that you have created, are waiting to join or are currently participating in.  Tap on the event to view the details of a event you created.  This opens the Event Administration page (Figure 3.24).   Enter the email address of the user you wish to invite in the Email Address Field.  Tap on the Send button to send the invitation to the user.

\clearpage
\section{Questionnaire Questions}
\label{sec:qquestions}

\subsection{Designing a questionnaire}
\label{sec:dquestions}

This section will describe how you can use the Template Builder (Figure 4.4) to add or modify questions to a template. To open the Template Builder, follow the steps described in section 4.2. 

Near the top of the Template Builder is the top toolbar with three buttons. The categories button    allows you to define categories of the template. The publish button   in the center allows you to publish the template to be visible as a questionnaire in the Community View page (Figure 3.13). The delete button   on the right allows you to delete the template.  Tap on each button will perform its respective action. 

The Template Builder contains the metadata of Name and Mutual Match Threshold. You can tap on the Name and Mutual Match Threshold Fields to update their values.  See section 4.4.1 for more details on the Mutual Match Threshold property.

\subsection{Minimum score for a match}
\label{sec:minscore}
As a designer, you can specify the threshold for a mutual match (see section 4.4). The threshold determines the minimum percentage compatibility between two questionnaire submissions that Matchmaker will use to determine a mutual match.  The suggested mutual match threshold for a good and balanced match is $75\%$ or above.

\subsection{Category Weighting}
\label{sec:categoryweight}
The set weights for question categories, navigate to the Template Builder page (Figure 4.4) as described in section 4.2.  Tap categories button    to open the Category Weight page (Figure 4.5).  On the Category Weight page you can tap a category to set the numeric value using the number pad.  The weight value is restricted between 0.0 and 2.0.

\subsection{Adding questions to a questionnaire}
\label{sec:addquestion}
On the bottom toolbar of the Template Builder (Figure 4.4), there are three buttons that create new questions - with each creating a different type of question. The three types of questions are  true-false  , multiple choice  , and multiple answer  . 

To add any of these types to the template, tap on the associated button.  This opens the Question Builder (Figure 4.6).  You can set the question in Question Field located under the Description header. Tap the text box area to set the text.  You can set the importance (see section 4.5.1),  category (see section 4.5.2), and responses (see section 4.5.3) of the question as well.

Once you are satisfied with your question, tap on the Save button at the bottom to save the question. 

\clearpage
\section{Questions}
\label{sec:questions}

\subsection{Question importance}
\label{sec:qimportance}
The importance of a question determines how it will be weighed when computing mutual matches.  Three stars means that the question will have higher weighting. Two stars, the default selection, means that the question will have normal weighting. One star means that the question has lower weighting.  To set the importance of the question you tap on the star icons   under the Importance header. 

\subsection{Question category}
\label{sec:qcategory}
The category of a question is used to organize the related questions.  The categories can then have a numeric factor applied to them to increase their relevance in the template.  This means that questions are weighted related to their own category.  To set the category of the question, tap on the desired colour icon  .  The select category will have a lined border around its edges  .

\subsection{Question responses}
\label{sec:qresponses}
The responses for a question change based on the question type.  The three types of questions are  true-false, multiple choice, and multiple answer.  The true-false question type has two possible responses, intended for binary responses.  The multiple choice question type allows you to select a single answer for a question from a list of provided options. The multiple answer question type allows you to select multiple answers for a single question from a list of provided options. 

The multiple answer and multiple choice question types allows you to add or remove additional responses. Tap the Plus button   under the Response header to add a new response. Tap the remove button  to the left of a response to remove it.

\subsection{Ordering questions}
\label{sec:qordering}
You can reorder the questions in a template on the Template Builder page (Figure 4.4). In the questions list, tap down on a question that you need to be moved and drag the question to its intended position. 

\subsection{Question match criteria}
\label{sec:matchcriterua}
\Friending{} compares the answers to questions in questionnaires to determine a numerical score to perform matches.  The question match criteria defines how many points are awarded based on the two provided answers.   The You column represents the chosen answer, and the Star Ratings ( ) represent the points award if the comparison answer has chosen this.  

On the Question Builder pages (Figure 4.7) for each of the question types, you can select the Matching Segment ( ) of the segmentation control.   

With the Match Segment enabled, you can set the scoring properties of each matching comparison.  Tap on the checkbox in the You column to see the scoring values of the question.  The Star Ratings ( ) represent the score value if comparison question had that option set.  The values are as such: an empty star is 0, half star is 0.5, and a full star is 1.  The comparison score from comparing two questions is equal to the summation of all score values, divided by the best possible score value for the question.

In Figure 4.7, the multiple-choice question has four options.  Option \#2 is currently selected, which results in the Star Ratings showing the points yielded if the comparison question had selected that answer.  The choice of Option \#2 yields a full star, and half stars (or 0.5 points) are yielded for Option \#3 and \#4.   

\clearpage
\section{Mutual Match}
\label{sec:matching}
\Friending{} employs a matchmaking algorithm that creates a mutual match between participants after submitting their questionnaires.  \Friending{} does this by computing the comparison score of $Q \in Questions$ and $A,B \in Answers-to-Q$:
\begin{equation}
(comparison score)_Question = COMP(A,B)*CAT(Q)*IMP(A)*IMP(A)*IMP(Q) 
\end{equation}

Where IMP is the importance rating from 1 to 3, CAT is the category weighting and COMP is the comparison as defined in section 4.5.5.   The comparison score for a questionnaire is then calculated based on the summation of the comparison score for all questions and their associated answers.
The compatibility score is computed from the comparison score of the questionnaire divided by the balanced comparison score for the questionnaire.  The balanced comparison score is $85\%$ of the best possible comparison score.

A mutual match is defined as a match between two participants who have a compatibility score equal to or above the minimum match threshold.  

\subsection{Group Matching}
\label{sec:groupmatch}
\Friending{} makes use a standard solution to the Stable Marriage Problem.

\subsection{Event Matching}
\label{sec:eventmatch}
\Friending{} makes use a standard solution to the Stable Roommate Problem.

Events are ...

Mixers are ..

\clearpage
\section{Troubleshooting \& Tips}
\label{sec:troubleshootandtips}

\subsection{Troubleshooting}
\label{sec:troubleshoot}
If you encounter difficulty while using \Friending{}, you should attempt each of the following steps until the problem is solved.

\begin{itemize}
\item Restart \Friending{}
\begin{enumerate}
\item \action{Close} the application
\item Wait a short period  after closing
\item \action{Open} \Friending{}
\end{enumerate}
\item Sign out then Sign In
\begin{enumerate}
\setNavigationMenuColor
\item \action{Tap} on the \appbutton{menu} button in the top left.
\item \action{Tap} on the \keys{Sign Out} button.
\item You will now be directed to \Screenshot{Sign In}
\item Follow Section ~\ref{sec:eventmatch} and sign in to \Friending{}
\setDefaultMenuColor
\end{enumerate}
\end{itemize}
\subsection{Tips}
\label{sec:tips}
\begin{enumerate}
\item Create a questionnaire before creating an event.
\item Group similar types of questions into categories.
\item Prototype a questionnaire by creating an event before publishing.
\end{enumerate}

If billing information on your account is incorrect, then you can contact \textbf{{billing@friending.no}}.

\clearpage
\section{Limitations}
\label{sec:limitations}

\Friending{} has limitations with respect to \abbrevation{G\&SD}.  Section ~\ref{sec:eventmatch} specifies the creation of events, specifically mixers, that ensure a match for all participants upon completion.  A mixer currently supports Heterosexual, Bisexual, Gay, and Lesbian matches.  With limited match support, the current version of \Friending{} does not support all \abbrevation{G\&SD}.  A future version may support matches providing more inclusivity for \abbrevation{G\&SD} individuals.

The following are limitations of \Friending{}:
\begin{enumerate}
\item You can only invite to an event by email address
\item You can only invite to a group by email address
\end{enumerate}

\end{document}
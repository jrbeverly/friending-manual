\section{Events}
\label{sec:events}


\subsection{Joining an event}
\label{sec:events-join}
The only way to join a event is to be invited to a event.  To view your invitations, \action{tap} on the \keys{Notifications} button on the \quicknav. On \Screenshot{Notifications} (Figure~\ref{fig:navigation}) you can view all notifications for your account.  \action{Tap} on a event invitation notification to open the \Screenshot{Event View} (Figure~\ref{fig:home}).  

On the \Screenshot{Event View} page you can tap on the \keys{Join} button to join the event. This will mark your intentions to join the event - it does not guarantee that you will successfully join the event. See section ~\ref{sec:events-wait} for more details.   You may not be able to join a event based on your gender or sexual orientation, see section ~\ref{sec:events-req} for more details. You will receive a notification of your intention to join the event, visibile in your \Screenshot{Notifications} (Figure~\ref{fig:navigation}).   When you have successfully joined the event, you will see the event on your \Screenshot{Home} (Figure~\ref{fig:home}), in \Screenshot{Events} (Figure~\ref{fig:events}), and a notification in your \Screenshot{Notifications} (Figure~\ref{fig:notifications}).  

If you no longer wish to join the event, you can cancel your pending request to join by revisiting the \Screenshot{Event View} page.  \action{Tap} the \keys{Pending} button to cancel your pending request (Figure~\ref{fig:joined}).  Once you have successfully joined a event, you cannot leave the event.

\subsection{Waiting to join an event}
\label{sec:events-wait}
When trying to join a event, if the event does not have sufficient participants for gender and sexual orientation matches, then you will be placed in a waiting list.  The waiting list is first come first serve.  As a event ensures that individuals will be matched with someone, you will remain in the waiting list until someone requests to join the event that has the potential to match with you. 

An example of this would be a heterosexual event.  This event requires an equal number of male and female participants to ensure that everyone will be matched with someone.  A male user requesting to join would be placed in a waiting list until a female user requests to join the event.  For a homosexual event, the event requires an even number of participants to ensure everyone is matched.  As such, a user requesting to join would be placed in the waiting list until another user requests to join the event.

When the event host start matching for the event, the waiting list will be cancelled and everyone on the list will be denied from joining the event.  If you do not successfully join the event, you will see a notification in your \Screenshot{Notifications}.

\subsection{Event Join Requirements}
\label{sec:events-req}
A event is defined with an orientation that defines how participants in a event are matched.  The currently defined orientations are Heterosexual, Bisexual, Gay, and Lesbian.  This means that a event with a Lesbian orientation would only accept you if you are female.  All participants in a event must match with someone in the event, as such the orientation restricts the participants to those that can match within the event.  See section ~\ref{sec:limitations} on the limitations of \Friending{} with respect to G\&SD.

\subsection{Viewing events}
\label{sec:events-view}
To view your events, \action{tap} on the \keys{Events} button on the \quicknav. This brings up the Events page (Figure~\ref{fig:event}). On the Events page (Figure~\ref{fig:events}) you can view all events that you have created, are waiting to join or are currently participating in.  \action{Tap} on the event to view the details of a event.  This opens the \Screenshot{Event} page (Figure~\ref{fig:event}).  

\subsection{Starting matching for events}
\label{sec:events-starting}
To host an event, \action{tap} on the \keys{Create Party} button on the \Screenshot{Home} (Figure~\ref{fig:home}). This opens the \Screenshot{Event Create} (Figure~\ref{fig:eventadmin}). 

On the Event Create page you can set the properties of the event.  \action{Enter} your changes to \textvar{Title}, \textvar{Capacity}, \textvar{Orientation}, and \textvar{Description} fields.  The \textcomp{Orientation} field is limited to Heterosexual, Bisexual, Gay, and Lesbian.  To set the questionnaire, see section ~\ref{sec:questionnaire-create}.  After you are satisfied with your changes, \action{tap} on the Create button to create your event.   You can now see your event on your \Screenshot{Home} (Figure~\ref{fig:home}) or on the Events page (Figure~\ref{fig:events}).

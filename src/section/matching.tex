\section{Mutual Match}
\label{sec:matching}
\Friending{} employs a matchmaking algorithm that creates a mutual match between participants after submitting their questionnaires.  \Friending{} does this by computing the comparison score of $Q \in Questions$ and $A,B \in Answers-to-Q$:
\begin{equation}
(comparison score)_Question = COMP(A,B)*CAT(Q)*IMP(A)*IMP(A)*IMP(Q) 
\end{equation}

Where IMP is the importance rating from 1 to 3, CAT is the category weighting and COMP is the comparison as defined in section 4.5.5.   The comparison score for a questionnaire is then calculated based on the summation of the comparison score for all questions and their associated answers.
The compatibility score is computed from the comparison score of the questionnaire divided by the balanced comparison score for the questionnaire.  The balanced comparison score is $85\%$ of the best possible comparison score.

A mutual match is defined as a match between two participants who have a compatibility score equal to or above the minimum match threshold.  

\subsection{Group Matching}
\label{sec:groupmatch}
\Friending{} makes use a standard solution to the Stable Marriage Problem.

\subsection{Event Matching}
\label{sec:eventmatch}
\Friending{} makes use a standard solution to the Stable Roommate Problem.

Events are.

Mixers are.
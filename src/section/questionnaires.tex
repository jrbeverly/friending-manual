\section{Questionnaires}
\label{sec:questionnaires}
\appscreenshot{\fileMatchingTypeOne}{Question Weights}{Set a weighting value for the answer options of a question}{fig:questionweight}
\appscreenshot{\fileMatchingTypeTwo}{Option Weights}{Set weighting values for each answer option}{fig:optionweight}
\appscreenshot{\fileQuestion}{Question}{Create a question for your questionnaire}{fig:question}
\appscreenshot{\fileQuestionnaire}{Questionnaire}{Construct your questionnaire for an event or group}{fig:questionnaire}
\appscreenshot{\fileQuestions}{Questions}{View the questions of your questionnaire}{fig:questions}
\appscreenshot{\fileQuestionnaires}{Questionnaires}{Manage your questionnaires}{fig:questionnaires}
\appscreenshot{\fileWeighting}{Weighting}{Weight answer options based on their importance}{fig:weighting}
\appscreenshot{\fileCategories}{Categories}{Use categories to group similar questions}{fig:categories}
\subsection{Filling out a questionnaire}
\label{sec:questionnaire-fill}
To fill out a questionnaire, tap on the questionnaire on the Group Home page (Figure 3.13). This opens the Questionnaire Overview page (Figure 3.14). 

On the Questionnaire Overview page you can view all questions in the questionnaire.  Each question is marked with a solid   to indicate an answered question or empty heart   if you have not.  All questions must be answered before you can submit.  If you leave the page before finishing the questionnaire, your previous answers will be saved. To answer a question tap on a question.  You can tap any question in the questionnaire, in any order.  After you are satisfied with your answers, tap on the Submit button to submit your questionnaire.  

\Friending{} currently supports three types of questions: true-false, multiple choice, and multiple answer.  Depending on the type of question, you will be presented with one of the three pages:

To fill out a response for any question type, tap on the circle next to the answer of your choice.
You fill in an answer for yourself and for your ideal match, under You and Them columns respectively.  After you are satisfied with your responses, tap on the Save button to save your responses for this question.

\subsection{Flagging important questions}
\label{sec:questionnaire-importance}
To rate the importance of the question shown in Figure 3.15, tap on the star rating  under the label Importance.  The star which you tap will determine the rating out of 3.  The more stars, the higher degree of  importance to you the question is.  This degree of importance will be factored into the matchmaking process.  The default rating is 1 star. 

\subsection{Deleting a draft questionnaire}
\label{sec:questionnaire-delete}
You cannot delete a questionnaire that has not been submitted.   You are only able to submit one set of answers for a questionnaire.  You can submit the questionnaire (see section 3.2.3), then delete the submitted questionnaire.

If you delete your account, the draft questionnaire information will be deleted.

\subsection{Viewing your submitted questionnaire}
\label{sec:questionnaire-submitted-view}
To view a submitted questionnaire, tap on the questionnaire on the Group Home page (Figure 3.13). This opens the Questionnaire Overview page (Figure 3.14).   On the Submitted Questionnaire Overview page (Figure 3.16) you can view all questions in the questionnaire.  To view a question, tap on a question.  You can tap any question in the questionnaire, in any order.  

\subsection{Updating your submitted questionnaire}
\label{sec:questionnaire-submitted-update}
To update a submitted questionnaire, tap on the questionnaire on the Group Home page (Figure 3.13). This opens the Submitted Questionnaire Overview page (Figure 3.16).  On the Questionnaire Overview page you can view all questions in the questionnaire.  To view a question tap on a question.  You can tap any question in the questionnaire, in any order.  

\Friending{} currently supports three types of questions: true-false, multiple choice, and multiple answer (Figure 3.15).  To update a response for any question type, tap on the circle next to the answer of your choice.  You fill in an answer for yourself and for your ideal match, under You and Them columns respectively.   To update the importance of a question, follow the process defined under section 3.2.4.

After you are satisfied with your updates, tap on the Save button to save your updates for this question. 

\subsection{Removing your submitted questionnaire}
\label{sec:questionnaire-submitted-delete}
To delete a submitted questionnaire, tap on the questionnaire on the Group Home page (Figure 3.13). On the Submitted Questionnaire Overview page (Figure 3.16) you can view all questions in the questionnaire.  To delete a submitted questionnaire, select the Delete button.  The submitted questionnaire cannot be recovered after deletion.

\subsection{Viewing your match notifications}
\label{sec:questionnaire-matches}

To view your match notification, tap on the Notifications button on the Navigation Menu (Figure 1.2). This brings up the Notifications page (Figure 3.17).  On the Notifications page you can view all notifications for your account.  Tap on the match found notification to open the Match Profile View page (Figure 3.18).  

On the Match Profile View page you can then contact the match through email.  At present \Friending{} only supports contacting matches through email.

\appscreenshot{\fileMatch}{Match}{View the profile of one of your match}{fig:match}
\appscreenshot{\fileNotifications}{Notifications}{Manage your incoming notifications}{fig:notifications}

\clearpage
\section{Questionnaire Designs}
\label{sec:admin}

\subsection{Creating a questionnaire}
\label{sec:questionnaire-create}
On the Party Create page (Figure 3.22) you set the questionnaire.  Tap the Select Template Field to open the Template Select page (Figure 3.23).  On the Template Select page you can view all templates you have created.  Tap on the template to select the template for your party.  You will be returned to the Party Create page with the template set.

\subsection{Sending invitiations}
\label{sec:questionnaire-invite}
To view your events, tap on the Events button on the Navigation Menu (Figure 1.2). This brings up the Events page (Figure 3.21).   On the Events page (Figure 3.21) you can view all events that you have created, are waiting to join or are currently participating in.  Tap on the event to view the details of a event you created.  This opens the Event Administration page (Figure 3.24).   Enter the email address of the user you wish to invite in the Email Address Field.  Tap on the Send button to send the invitation to the user.

\clearpage
\section{Questionnaire Questions}
\label{sec:qquestions}

\subsection{Designing a questionnaire}
\label{sec:dquestions}

This section will describe how you can use the Template Builder (Figure 4.4) to add or modify questions to a template. To open the Template Builder, follow the steps described in section 4.2. 

Near the top of the Template Builder is the top toolbar with three buttons. The categories button    allows you to define categories of the template. The publish button   in the center allows you to publish the template to be visible as a questionnaire in the Community View page (Figure 3.13). The delete button   on the right allows you to delete the template.  Tap on each button will perform its respective action. 

The Template Builder contains the metadata of Name and Mutual Match Threshold. You can tap on the Name and Mutual Match Threshold Fields to update their values.  See section 4.4.1 for more details on the Mutual Match Threshold property.

\subsection{Minimum score for a match}
\label{sec:minscore}
As a designer, you can specify the threshold for a mutual match (see section 4.4). The threshold determines the minimum percentage compatibility between two questionnaire submissions that Matchmaker will use to determine a mutual match.  The suggested mutual match threshold for a good and balanced match is $75\%$ or above.

\subsection{Category Weighting}
\label{sec:categoryweight}
The set weights for question categories, navigate to the Template Builder page (Figure 4.4) as described in section 4.2.  Tap categories button    to open the Category Weight page (Figure 4.5).  On the Category Weight page you can tap a category to set the numeric value using the number pad.  The weight value is restricted between 0.0 and 2.0.

\subsection{Adding questions to a questionnaire}
\label{sec:addquestion}
On the bottom toolbar of the Template Builder (Figure 4.4), there are three buttons that create new questions - with each creating a different type of question. The three types of questions are  true-false  , multiple choice  , and multiple answer  . 

To add any of these types to the template, tap on the associated button.  This opens the Question Builder (Figure 4.6).  You can set the question in Question Field located under the Description header. Tap the text box area to set the text.  You can set the importance (see section 4.5.1),  category (see section 4.5.2), and responses (see section 4.5.3) of the question as well.

Once you are satisfied with your question, tap on the Save button at the bottom to save the question. 

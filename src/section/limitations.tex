\section{Limitations}
\label{sec:limitations}

%% Sexuality
%%
\subsection{Sexuality}
\label{sec:limit_sexuality}

\Friending{} currently provides support for Heterosexual, Gay, and Lesbian matches. The current matching algorithm can only handle bipartite matching, resulting in minimal support for \abbrevation{G\&SD}. The matching algorithm is defined in Section ~\ref{sec:matching}, addressing the matchmaking requirements. With the current algorithm, individuals that are outside of the gender binary are unable to participant in a mixer, due to its unique matchmaking requirements. A mixer is a type of event that ensures a match for all participants upon completion.

With limited match support, the current version of \Friending{} cannot be considered to fully support \abbrevation{G\&SD}. A future release may provide better matching support.

%% Customization
%%
\subsection{Customization}
\label{sec:limit_questions}

User customization is limited to images, specifically the banner photos. \Friending{} has made the choice to limit user customization for the sake of uniformity, and encouraging simplicity in design. The \textbf{Group and Event Guidelines} document encourages users to attract potential participants using their photo. A captivating photo is great way to encourage others to join the community.

The \Friending{} team deems this limitation essential to ensuring the application facilitates real-world action.

%% Mobile Only
%%
\subsection{Mobile Only}
\label{sec:limit_mobile}

\Friending{} is built with a 'Mobile First' philosophy, driven by a vision for users to always have a chance to connect with their potential matches. This philosophy has trade-offs, which are listed below. These limitations are not ordered in any way.

\begin{enumerate}
\item Emphasis on email addresses and push notifications as primary forms of communication
\item Billing and account services accessible only through the mobile application
\item Limited support for all devices
\item Limited screen resolution restricting \abbrevation{UX} options
\end{enumerate}
\section{Conventions}
\label{sec:conventions}

\subsection{User assumptions}
\label{sec:assumptions}
You, the user of \Friending{}, are assumed to know the basics of touch-enabled applications.  If not, please refer to a user’s manual of your mobile device.  \Friending{} is only accessible through an internet-enabled mobile device.

\subsection{Use Cases}
\label{sec:usecases}
A use case is.  All the manipulations in a use case are based on the assumption that \Friending{} is running, and you are signed in to your account unless otherwise specified.  For Sections ~\ref{sec:signup} and ~\ref{sec:signin}, you are assumed to not be signed in.  

\subsection{Notational conventions}
\label{sec:notational}
The text conventions below are used in this manual:

\begin{itemize}
\item \textapp{Computer Modern Sans Serif} is used for program names and file names.
\item Computer Modern Roman is used for normal text.
\item \textuser{Times New Roman Italics} is used for key users of the software.
\item \textmath{Palatino} is used for mathematical formulas.
\item \textvar{Palatino Small Capitals} is used for key name variables 
\item \textcomp{Avant Garde} is used for key components of the software.
\item \textui{Avant Garde Bold} is used for user interface components of the software.
\item \textaction{Computer Modern Typewriter} is used for physical actions taken by the user when interacting with the user interface of the software.
\end{itemize}

\subsection{Visual conventions}
\label{sec:visual}
The images below are used throughout the manual:

\begin{itemize}
\item \appbutton{editting} is used to indicate a field that can be toggled for editing.
\item \appbutton{calendar} is used to indicate a field that is editable and uses a calendar widget.
\item \appbutton{menu} is used to indicate a pull out menu.
\item is used to indicate the pages of a carousel menu.
\item \setDefaultMenuColor \keys{Sample}  is used to denote buttons of the user interface.
\item \setNavigationMenuColor \keys{Sample} is used to denote menu buttons of the user interface.
\end{itemize}
\setDefaultMenuColor

\subsection{Glossary of terms}
\label{sec:glossary}
The terms below are used through the manual:

\begin{itemize}
\gterm{\Friending{}}{The name of the application.}
\gterm{User}{The person who uses \Friending{}, addressed by “you”.}
\gterm{Member}{A user who joins a group.}
\gterm{Participant}{A user who fills out a response to a questionnaire.}

\gterm{Internet-Enabled Mobile Device}{A small computing device capable of accessible the internet.}
\gterm{One Time Password}{A password that is valid for a single sign on.}
\gterm{Approachable}{The trait of being approachable.}

\gterm{Fee}{A charge for \Friending{} services.}
\gterm{Billed}{A statement of money owed for \Friending{} services.}

\gterm{Profile}{The outline of "you", the user.}
\gterm{Event}{A one time event where multiple participants fill out questionnaires to receive matches with other participants in the event}
\gterm{Event Host}{A participant who initiates a event}
\gterm{Group}{A collection of members and questionnaires administered by a Group Administrator.}
\gterm{Group Administrator}{The administrator for a group that has questionnaires and members.}
\gterm{Questionnaire}{A set of questions to be answered by the participant.}
\gterm{Notification}{An email sent to a user.}

\gterm{Match}{A pairing of exactly two participants.}
\gterm{Mutual Match}{A match between two participants who have a compatibility score equal to or above the minimum match threshold.} 
\gterm{Minimum Match Threshold}{A percentage point at which the compatibility score between two participants is sufficient for a mutual match.}
\gterm{Compatibility Score}{The percentage compatibility between two participants.}
\gterm{Comparison Score}{The numeric value representing the similarity between two values.}
\gterm{Balanced Comparison Score}{A numeric value equalling $85\%$ of the best possible comparison score for a questionnaire.}
\gterm{Question Match Criteria}{The criteria determining the comparison score between two question answers.}

\gterm{Query}{A request for information from \Friending{}.}

\gterm{Number Pad}{A grid of numbers used for inputting numeric values.}

\gterm{Heterosexual}{Is romantic attraction, sexual attraction or sexual behavior between persons of the opposite sex or gender.}
\gterm{Bisexual}{Is romantic love or sexual attraction toward both males and females}
\gterm{Gay}{A male homosexual a male who experiences romantic love or sexual attraction to other males.}
\gterm{Lesbian}{A female homosexual a female who experiences romantic love or sexual attraction to other females.}
\end{itemize}

\subsection{Abbreviations}
\label{sec:abbreviations}
\begin{itemize}
\item G\&SD - Gender and Sexual Diversities
\item GUI - Graphic User Interface
\item SMS - Short Messaging Service
\end{itemize}

\subsection{Basic User interface and user interactions goals}
\label{sec:goals}
\Friending{} is inspired by mail-in oriented dating services that appeared in the early stages of "online" dating.  These computer dating services operate by having you fill out a paper questionnaire which will be mailed in with a nominal fee.  The questionnaires are geared towards people seeking a date.  

\Friending{} makes use of the internet, removing the slow and sometimes unreliable mail service.  You can fill out, submit and view your responses to questions faster than using postal mail.  Additionally \Friending{} includes more communal oriented features than the early concepts, such as enabling you to host your own events with questionnaires.

The basic goals with the user interface of \Friending{} is the following:
\begin{itemize}
\item Designing a template should be approachable due to simplified rating systems. 
\item Each view should have an obvious feature and method of navigation.
\item There is no need to require a time commitment; \Friending{} works in the background.
\end{itemize}

\subsection{Organization of this manual}
\label{sec:organization}
The remainder of this manual is organized based on use cases.  All Sections assume that you are already logged into \Friending{} with the exception of Sections ~\ref{sec:signup} and ~\ref{sec:signin}. This manual also contains troubleshooting in Section ~\ref{sec:troubleshootandtips} and  gives the limitations of the current version of \Friending{} in Section ~\ref{sec:limitations}.